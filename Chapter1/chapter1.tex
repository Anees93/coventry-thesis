%!TEX root = ../thesis.tex
%*******************************************************************************
%*********************************** First Chapter *****************************
%*******************************************************************************

\chapter{Introduction to CULTT}  %Title of the First Chapter

\ifpdf
    \graphicspath{{Chapter1/Figs/Raster/}{Chapter1/Figs/PDF/}{Chapter1/Figs/}}
\else
    \graphicspath{{Chapter1/Figs/Vector/}{Chapter1/Figs/}}
\fi

%********************************** First Section ******************************
\section{What is CULTT?} %Section - 1.1 

The Coventry University \LaTeX{} Thesis Template (CULTT) is designed as a starting point for writing and formatting a Coventry University (CU) Thesis prior to submission for examination and printing. This template is not official in anyway and it is up to the thesis author to ensure that the format, requirements and contents of the thesis are correct. However, this template should prove useful for those using \LaTeX{} to prepare their CU Thesis. CULTT was originally put together as a CU Thesis template for use on ShareLaTeX, the online service for producing documents with \LaTeX{}. ShareLaTeX also supports collaborative working, see \url{https://www.sharelatex.com/} for futher information. Another online \LaTeX{} editor and collaboration service is Overleaf, \url{https://www.overleaf.com/}.

If you are not using \LaTeX{} you will not need CULTT. However, the use of \LaTeX{} for producing all academic documents (not only a thesis) is recommended. Although \LaTeX{} may appear strange to those who normally use word processing software, the benefits are immense once you get use to it. ShareLaTeX has some great documentation to help get started, see \url{https://www.sharelatex.com/learn/}. The \LaTeX{} Project website is at \url{https://www.latex-project.org/}. If not using an online \LaTeX{} service then install a \LaTeX{} environment onto your computer. MiKTeX is a good package which can be installed on Windows, Mac, and Linux, see \url{https://miktex.org/} (also available from the CU Cloudpaging Player Apps service). MiKTeX includes the TeXworks editor, \url{https://www.tug.org/texworks/}. A \LaTeX{} for Beginners document is available at \url{http://www.docs.is.ed.ac.uk/skills/documents/3722/3722-2014.pdf}.

%********************************** Second Section  ****************************
\section{Using CULTT} %Section - 1.2

CULTT can be used on a local computer or uploaded to a online \LaTeX{} service. The thesis template is divided into sub-folders for ease of organisation. All of the textual content in the various \LaTeX{} files (e.g. this Introduction chapter) will be replaced with the authors own text. The content is stored in the \textit{tex} files can brought together in the top level \textit{thesis.tex}.

There are a few requirements for a correct CU Thesis submission. To see those requirements first visit the CU website for the Academic Regulations, \url{http://www.coventry.ac.uk/life-on-campus/the-university/key-information/registry/academic-regulations/} and download section 8 to read \textbf{8.12 The higher degree thesis}. 

The CU Student Portal, accessed via the Portals link on the CU website, has Links to relevant documents in the Doctoral College section (under Study at CU). See the Preparing for your PRP and Viva  section to find the Thesis Information PDF (called Thesis Requirements.pdf). Further CU help on Thesis writing can be obtained from the Centre for Academic Writing (CAW), \url{https://cawbookings.coventry.ac.uk/}. The CU Harvard Referencing Style information is available on the website \url{https://www.cuguide-toharvard.info/}. The root document for CULTT is thesis.tex which includes other tex files as required. This allows the Thesis to be divided into smaller files for better management and safety. Working on smaller sections of the Thesis reduces the chances of accidentally destroying large chunks of work. (Always take regular backups of all work produced.)